\documentclass[11pt]{article}
\usepackage{geometry}
\usepackage{helvet}
\usepackage{setspace}
\usepackage{cite}

\geometry{
    a4paper,
    left=0.5in,
    right=0.5in,
    top=0.5in,
    bottom=0.5in,
}
\renewcommand{\familydefault}{\sfdefault}
\singlespacing
\setlength{\parskip}{1em}

\title{Insert Your Project Title Here}
\author{Your Name Here \\ Affiliation or Institution Here}
\date{}

\begin{document}
\maketitle
\thispagestyle{empty}
\pagestyle{empty}

\noindent \textbf{1. Scientific Approach}

Provide an overview of your scientific approach. Detail the background, rationale, and methodology in a concise manner, highlighting how it is innovative or groundbreaking in the field.

\noindent \textbf{2. AI/Technology Approach}

Describe the specific artificial intelligence or technological methods you will utilize in your project. Explain how these approaches are suited to the goals of your project and their advantage over existing methods.

\noindent \textbf{3. Relevance to Healthy Aging and/or AD/ADRD}

Clearly articulate how your project is relevant to the fields of healthy aging, Alzheimer's Disease (AD), or Alzheimer's Disease-Related Dementias (ADRD). Your explanation should draw direct connections to the impact this project can have in these fields.

\noindent \textbf{4. Aims and Expected Outcomes}

List the specific aims of your project and the expected outcomes. Be sure to clarify the significance of these outcomes and their potential impact on the field or in practice.

\noindent \textbf{5. Plans or Potential for Translation or Commercialization of Project Deliverables}

Discuss any plans or the potential pathways for the translation of your project's results into practical applications or for commercialization. Include considerations on scalability, market need, and potential barriers to implementation.

\begin{spacing}{0.5}
\bibliographystyle{ieeetr} % or another suitable bibliographic style.
\bibliography{references} % references.bib contains your bibliographic information.
\end{spacing}

\end{document}
